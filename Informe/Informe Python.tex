%package list
\documentclass{article}
\usepackage[top=3cm, bottom=3cm, outer=3cm, inner=3cm]{geometry}
\usepackage{multicol}
\usepackage{graphicx}
\usepackage{url}
%\usepackage{cite}
\usepackage{hyperref}
\usepackage{array}
%\usepackage{multicol}
\newcolumntype{x}[1]{>{\centering\arraybackslash\hspace{0pt}}p{#1}}
\usepackage{natbib}
\usepackage{pdfpages}
\usepackage{multirow}
\usepackage[normalem]{ulem}
\useunder{\uline}{\ul}{}
\usepackage{svg}
\usepackage{xcolor}
\usepackage{listings}
\lstdefinestyle{ascii-tree}{
	literate={├}{|}1 {─}{--}1 {└}{+}1 
}
\lstset{basicstyle=\ttfamily,
	showstringspaces=false,
	commentstyle=\color{red},
	keywordstyle=\color{blue}
}
%\usepackage{booktabs}
\usepackage{caption}
\usepackage{subcaption}
\usepackage{float}
\usepackage{array}

\newcolumntype{M}[1]{>{\centering\arraybackslash}m{#1}}
\newcolumntype{N}{@{}m{0pt}@{}}


%%%%%%%%%%%%%%%%%%%%%%%%%%%%%%%%%%%%%%%%%%%%%%%%%%%%%%%%%%%%%%%%%%%%%%%%%%%%
%%%%%%%%%%%%%%%%%%%%%%%%%%%%%%%%%%%%%%%%%%%%%%%%%%%%%%%%%%%%%%%%%%%%%%%%%%%%
\newcommand{\itemEmail}{wchoquehuancab@unsa.edu.pe}
\newcommand{\itemStudent}{William Herderson Choquehuanca Berna}
\newcommand{\itemCourse}{Laboratorio de P.Web}
\newcommand{\itemCourseCode}{20233469}
\newcommand{\itemSemester}{III}
\newcommand{\itemUniversity}{Universidad Nacional de San Agustín de Arequipa}
\newcommand{\itemFaculty}{Facultad de Ingeniería de Producción y Servicios}
\newcommand{\itemDepartment}{Departamento Académico de Ingeniería de Sistemas e Informática}
\newcommand{\itemSchool}{Escuela Profesional de Ingeniería de Sistemas}
\newcommand{\itemAcademic}{2024 - A}
\newcommand{\itemInput}{Del 20 de mayo 2024}
\newcommand{\itemOutput}{Al 24 de mayo 2024}
\newcommand{\itemPracticeNumber}{05}
\newcommand{\itemTheme}{Python}
%%%%%%%%%%%%%%%%%%%%%%%%%%%%%%%%%%%%%%%%%%%%%%%%%%%%%%%%%%%%%%%%%%%%%%%%%%%%
%%%%%%%%%%%%%%%%%%%%%%%%%%%%%%%%%%%%%%%%%%%%%%%%%%%%%%%%%%%%%%%%%%%%%%%%%%%%

\usepackage[english,spanish]{babel}
\usepackage[utf8]{inputenc}
\AtBeginDocument{\selectlanguage{spanish}}
\renewcommand{\figurename}{Figura}
\renewcommand{\refname}{Referencias}
\renewcommand{\tablename}{Tabla} %esto no funciona cuando se usa babel
\AtBeginDocument{%
	\renewcommand\tablename{Tabla}
}

\usepackage{fancyhdr}
\pagestyle{fancy}
\fancyhf{}
\setlength{\headheight}{30pt}
\renewcommand{\headrulewidth}{1pt}
\renewcommand{\footrulewidth}{1pt}
\fancyhead[L]{\raisebox{-0.2\height}{\includegraphics[width=3cm]{img/logo_episunsa.png}}}
\fancyhead[C]{\fontsize{7}{7}\selectfont	\itemUniversity \\ \itemFaculty \\ \itemDepartment \\ \itemSchool \\ \textbf{\itemCourse}}
\fancyhead[R]{\raisebox{-0.2\height}{\includegraphics[width=1.2cm]{img/logo_abet}}}
\fancyfoot[L]{William Choquehuanca Berna}
\fancyfoot[C]{\itemCourse}
\fancyfoot[R]{Página \thepage}

% para el codigo fuente
\usepackage{listings}
\usepackage{color, colortbl}
\definecolor{dkgreen}{rgb}{0,0.6,0}
\definecolor{gray}{rgb}{0.5,0.5,0.5}
\definecolor{mauve}{rgb}{0.58,0,0.82}
\definecolor{codebackground}{rgb}{0.95, 0.95, 0.92}
\definecolor{tablebackground}{rgb}{0.8, 0, 0}

\lstset{frame=tb,
	language=bash,
	aboveskip=3mm,
	belowskip=3mm,
	showstringspaces=false,
	columns=flexible,
	basicstyle={\small\ttfamily},
	numbers=none,
	numberstyle=\tiny\color{gray},
	keywordstyle=\color{blue},
	commentstyle=\color{dkgreen},
	stringstyle=\color{mauve},
	breaklines=true,
	breakatwhitespace=true,
	tabsize=3,
	backgroundcolor= \color{codebackground},
}

\begin{document}
	
	\vspace*{10px}
	
	\begin{center}	
		\fontsize{17}{17} \textbf{ Informe de Laboratorio \itemPracticeNumber}
	\end{center}
	\centerline{\textbf{\Large Tema: \itemTheme}}
	%\vspace*{0.5cm}	
	
	\begin{flushright}
		\begin{tabular}{|M{2.5cm}|N|}
			\hline 
			\rowcolor{tablebackground}
			\color{white} \textbf{Nota}  \\
			\hline 
			\\[30pt]
			\hline 			
		\end{tabular}
	\end{flushright}	
	
	\begin{table}[H]
		\begin{tabular}{|x{4.7cm}|x{4.8cm}|x{4.8cm}|}
			\hline 
			\rowcolor{tablebackground}
			\color{white} \textbf{Estudiantes} & \color{white}\textbf{Escuela}  & \color{white}\textbf{Asignatura}   \\
			\hline 
			{\itemStudent \par \itemEmail} & \itemSchool & {\itemCourse \par Semestre: \itemSemester \par Código: \itemCourseCode}     \\
			\hline 			
		\end{tabular}
	\end{table}		
	
	\begin{table}[H]
		\begin{tabular}{|x{4.7cm}|x{4.8cm}|x{4.8cm}|}
			\hline 
			\rowcolor{tablebackground}
			\color{white}\textbf{Laboratorio} & \color{white}\textbf{Tema}  & \color{white}\textbf{Duración}   \\
			\hline 
			\itemPracticeNumber & \itemTheme & 04 horas   \\
			\hline 
		\end{tabular}
	\end{table}
	
	\begin{table}[H]
		\begin{tabular}{|x{4.7cm}|x{4.8cm}|x{4.8cm}|}
			\hline 
			\rowcolor{tablebackground}
			\color{white}\textbf{Semestre académico} & \color{white}\textbf{Fecha de inicio}  & \color{white}\textbf{Fecha de entrega}   \\
			\hline 
			\itemAcademic & \itemInput &  \itemOutput  \\
			\hline 
		\end{tabular}
	\end{table}
	
	\section{Actividades}
	\begin{itemize}		
		\item 1. Instale Python!\\
		2. Cree su entorno de trabajo:\\
		\hspace*{1cm}	mkdir lab04 \\
		\hspace*{1cm}	cd lab04\\
		\hspace*{1cm}	virtualenv -p python3 .\\
		\hspace*{1cm}	mkdir src\\
		\hspace*{1cm}	cd src\\
		\hspace*{1cm}	git init .\\
		3. Active el entorno virtual\\
		\hspace*{1cm}	source ../bin/activate\\
		4. Ahora podrá instalar las bibliotecas que necesite usando pip\\
		\hspace*{1cm}	pip install pygame\\
		5. Cuando quiera terminar de usar el entorno virtual ejecute\\
		\hspace*{1cm}	deactivate
		
	\end{itemize}
	
	\section{Ejercicios Propuestos}
	\begin{itemize}	
	
		\item Implemente los métodos de la clase Picture. Se recomienda que implemente la clase picture por etapas, probando realizar los dibujos que se muestran en la siguiente preguntas.
		Con el código proporcionado usted dispondrá de varios objetos de tipo Picture para poder realizar su tarea
		
		
		\item Usando únicamente los métodos de los objetos de la clase Picture dibuje las siguientes figuras (invoque a draw)
		\begin{figure}[H]
			\centering
			\includegraphics[width=0.65\textwidth, keepaspectratio]{img/figura1}
			\includegraphics[width=0.65\textwidth, keepaspectratio]{img/figura2}
		\end{figure}
	\end{itemize}
	
	\section{Equipos, materiales y temas utilizados}
	\begin{itemize}
		\item Sistema operativo de 64 bits, procesador basado en x64.
		\item Latex. 
		\item git version 2.41.0.windows.1
		\item Lenguaje Python.
		\item IDE Visual Sudio Code.
	\end{itemize}
	\section{URL Github, Video}
	\begin{itemize}
		\item URL del Repositorio GitHub.
		\item \url{https://github.com/WilliamLawrence25/PWeb2/tree/main/Lab5}
		\item URL para el video flipgrid.
		\item \url{https://flip.com/s/QRG7DhggB2kt}	
	
	\end{itemize}
	\clearpage	
	
	\section{Capturas de los ejercicios propuestos}
	
	\subsection{Ejercicio 1}
	\begin{figure}[H]
		\centering
		\
		\includegraphics[width=1.0\textwidth, keepaspectratio]{img/ejercicio1a}
		\includegraphics[width=1.0\textwidth, keepaspectratio]{img/ejercicio1b}
	\end{figure}
	
	\subsection{Ejercicio 2a}
	\begin{itemize}
		\item Python
	\end{itemize}
	\begin{figure}[H]
		\centering
		\includegraphics[width=1.0\textwidth, keepaspectratio]{img/ejercicio2a}
	\end{figure}
	\begin{itemize}
		\item Tablero
	\end{itemize}
	\begin{figure}[H]
		\centering
		\includegraphics[width=1.0\textwidth, keepaspectratio]{img/ejercicio2aa}
	\end{figure}
	
	\subsection{Ejercicio 2b}
	\begin{itemize}
		\item Python
	\end{itemize}
	\begin{figure}[H]
		\centering
		\includegraphics[width=1.0\textwidth, keepaspectratio]{img/ejercicio2b}
	\end{figure}
	\begin{itemize}
		\item Tablero
	\end{itemize}
	\begin{figure}[H]
		\centering
		\includegraphics[width=1.0\textwidth, keepaspectratio]{img/ejercicio2bb}
	\end{figure}
	
	\subsection{Ejercicio 2c}
	\begin{itemize}
		\item Python
	\end{itemize}
	\begin{figure}[H]
		\centering
		\includegraphics[width=1.0\textwidth, keepaspectratio]{img/ejercicio2c}
	\end{figure}
	\begin{itemize}
		\item Tablero
	\end{itemize}
	\begin{figure}[H]
		\centering
		\includegraphics[width=1.0\textwidth, keepaspectratio]{img/ejercicio2cc}
	\end{figure}
	
	\subsection{Ejercicio 2d}
	\begin{itemize}
		\item Python
	\end{itemize}
	\begin{figure}[H]
		\centering
		\includegraphics[width=1.0\textwidth, keepaspectratio]{img/ejercicio2d}
	\end{figure}
	\begin{itemize}
		\item Tablero
	\end{itemize}
	\begin{figure}[H]
		\centering
		\includegraphics[width=1.0\textwidth, keepaspectratio]{img/ejercicio2dd}
	\end{figure}
	
	\subsection{Ejercicio 2e}
	\begin{itemize}
		\item Python
	\end{itemize}
	\begin{figure}[H]
		\centering
		\includegraphics[width=1.0\textwidth, keepaspectratio]{img/ejercicio2e}
	\end{figure}
	\begin{itemize}
		\item Tablero
	\end{itemize}
	\begin{figure}[H]
		\centering
		\includegraphics[width=1.0\textwidth, keepaspectratio]{img/ejercicio2ee}
	\end{figure}
	
	\subsection{Ejercicio 2f}
	\begin{itemize}
		\item Python
	\end{itemize}
	\begin{figure}[H]
		\centering
		\includegraphics[width=1.0\textwidth, keepaspectratio]{img/ejercicio2f}
	\end{figure}
	\begin{itemize}
		\item Tablero
	\end{itemize}
	\begin{figure}[H]
		\centering
		\includegraphics[width=1.0\textwidth, keepaspectratio]{img/ejercicio2ff}
	\end{figure}
	
	\subsection{Ejercicio 2g}
	\begin{itemize}
		\item Python
	\end{itemize}
	\begin{figure}[H]
		\centering
		\includegraphics[width=1.0\textwidth, keepaspectratio]{img/ejercicio2g}
	\end{figure}
	\begin{itemize}
		\item Tablero
	\end{itemize}
	\begin{figure}[H]
		\centering
		\includegraphics[width=1.0\textwidth, keepaspectratio]{img/ejercicio2gg}
	\end{figure}
	
	
	
	
	\clearpage
	
	\section{Referencias}
	\begin{itemize}			
		\item \url{https://github.com/rescobedoq/pw2/tree/main/labs/lab04/Tarea-del-Ajedrez}
	\end{itemize}	
	
	%\clearpage
	%\bibliographystyle{apalike}
	%\bibliographystyle{IEEEtranN}
	%\bibliography{bibliography}
	
\end{document}